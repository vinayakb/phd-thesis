\chapter{Conclusions and Future Work}
\label{ch:conclusions-and-future-work}

This thesis presented Hyracks, a high-performance parallel dataflow runtime layer, and Algebricks, a parallel query compilation layer, both of which underpin the AsterixDB stack. Hyracks in conjunction with Algebricks also support quite a few other query processing systems. In addition to AsterixDB, described in Chapter~\ref{ch:asterixdb}, Hivesterix and Apache VXQuery are two other query processing engines that have been built on top of Hyracks + Algebricks. Hyracks has also served as a research platform for building specialized Big Data processing systems. Pregelix~\cite{pregelix} is a clone of the Pregel~\cite{Pregel} programming model for performing graph analytics over large distributed graphs, using a dataflow approach rather than a message-passing approach to solve the problem. Hyracks has also been used to implement a highly scalable version of the Batch Gradient Descent Algorithm for machine learning~\cite{DBLP:journals/corr/abs-1303-3517}.

Since Hyracks was first implemented and tested in 2011~\cite{hyracks}, we had the opportunity to test the software at scale (thanks to Yahoo! Research collegues who gave us access to their 180-node research cluster). The inter-process communication layers in Hyracks were re-architected based on lessons learned while running at that scale. Hyracks has also been used by outside groups as a research system. Researchers at Rice University used Hyracks to prototype Online Aggregation~\cite{DBLP:journals/pvldb/PansareBJC11}. A research group at UCSD working on Lineage Analysis instrumented Hyracks with their algorithms~\cite{Logothetis:2013:SLC:2523616.2523619}.

Hyracks still has a lot of room for improvement. For example, Hyracks implements a fairly simple form of fault-tolerance where a failed part of a job is re-executed until it succeeds. More sophisticated strategies, perhaps one that checkpoints operators periodically, could be used to reduce the amount of time spent in reruns. Furthermore, a cost-based analysis could be used to decide on the best operators to checkpoint in order to reduce the cost of the entire job. Automatic resource management is one other area that remains very open in the Hyracks platform. Distributing compute resources among concurrent queries is an important field of study that is not dealt with in this thesis. Hyracks uses a manual approach to resource management, where it is the programmer's or query compiler's responsibility to specify the amount of parallelism to use when running a job. In the future, Hyracks could benefit from an automatic resource management architecture, so that jobs could be optimized using various policies without the over (or under) consumption of available resources.

The Algebricks compilation framework has proven to be a valuable resource in building parallel query compilers, including several in addition to the AQL compiler in AsterixDB. As mentioned earlier, Algebricks has been used to build both Hivesterix (a Hive clone on Hyracks) and Apache VXQuery. Based on our experience building these query compilers, we have encountered some areas that could be explored in the future. Algebricks currently contains a rule-based optimizer. Cost-based optimization inside the Algebricks framework would be a welcome addition, and it would then benefit all the compilers built on top of Algebricks. In fact, the Database Research Group at IIT, Bombay has been working to embed their cost-based optimizer into the platform. Algebricks could also be extended to model iterative processing by perhaps adding an operator similar to the Alpha Operator~\cite{DBLP:journals/tse/Agrawal88}. Native iterative processing support would in turn allow Algebricks to be used to model systems like Pregelix and IMRU (both of which are currently hand-coded on top of Hyracks) and thus benefit from having an optimization step instead of relying on hand-coded plans as they do today.

AsterixDB, a project that originally started as a collaboration between the three Universities of California, at Irvine, Riverside, and San Diego, is now incubating at the Apache Software Foundation. Hyracks and Algebricks are now distributed as part of the AsterixDB project.
